\section{Introduction}
\vspace{0.2in}
\hspace*{0.16in}

\section{Related Work}
\vspace{0.2in}
\hspace*{0.16in}
Blood cell segmentation is the extraction of different blood cells from microscopic images. Blood cell counting is the process of counting the detected blood cells after segmentation. Many researchers have implemented methods for segmentation and counting of blood cells.\\

Kimbahune et al \textsuperscript{\cite{kimbahune2011blood}} have developed a method for segmenting and counting red blood cells (RBC) and white blood cells (WBC).
segmentation is done using Pulse-Coupled Neural Network (PCNN) and square tracing algorithm for countour tracing after de-noising it with PCNN combined with median filter, the counting is performed by scaning the image and using edge detection methods as square tracing algorithm. this method gave good results compared to state of art methods.\\
%this kimbahune article they didn't give any info about the database or the experimental results 

Bhavnani et al \textsuperscript{\cite{bhavnani2016segmentation}} have developed a method for segmenting and counting RBC (red blood cells), WBC (White blood cells) and platelets which is also called complete blood count (CBC), by using Otsu’s thresholding and morphological operations as a segmentation method, and for counting the are preforming a comparison between two methods: the watershed algorithm and Circular Hough Transform. The model takes an RGB image as an input apply some processing steps then uses Otsu's Thresholding to extract RBC and WBC separately with different threshold values then apply the two algorithm to compare the results, the model has no image size constraint because it's based only on image processing techniques and needs a small database to select the threshold values for RBC and WBC in this article they used 20 images. In the Experiment phase they used ALLIDB Database which contains 108 images with 1712x1368 and 2592x1944 resolution.The CHT method is the best in terms of accuracy with 92.67\% but it has some weaknesses with overlapping cells and morphological abnormalities. In the other side the watershed method which is a little bit adapted with overlapping and touching cells had an accuracy of 91.07\%.\\

Carlos X. Hern{\'{a}}ndez et al \textsuperscript{\cite{DBLP:journals/corr/abs-1802-10548}} have implemented a convolutional neural network (CNN) using a feature pyramid network (FPN) combined with a VGG style neural network for segmenting and counting of cells in a given microscopy image.
The dataset they used is BBBC005 from Broad Institute's Bioimage Benchmark Collection, which consists of 9600 images and each image is 696x520 pixels but they were scaled down to 256x192 for the purposes of their experiment.

\newpage

Out of the total 9600 images only 600 of the images which have a corresponding mask were used for the FPN training. And 100 of those were used for fast prototyping and a standard of 80-20 train/test split for the final models.
On the other hand, the full 9600 images were used for the VGG network.
This approach achieved a relatively good accuracy of 95\% but with some failure cases such as:

\begin{itemize}
  \item High cell overlap
  \item Irregular cell shapes
  \item bad focal planes.
\end{itemize}

Tran, Thanh and Minh et al \textsuperscript{\cite{DBLP:journals/corr/abs-1802-10548}} have developed a method for segmenting and counting RBC and WBCs by using the SegNet model with weights from a pre-trained VGG-16 model, for the counting they first apply Distance transform with 4 different distance metrics, then they apply binary dilation. At the End, they apply the connected component labeling algorithm to count the number of separated cells in images mask. the model had an accuracy of 89\%.

K. Sudha and P. Geetha \textsuperscript{\cite{SUDHA2020639}} have developed a two stage framework which will segment the leukocytes (a type of WBC) with an edge strength-based Grabcut method as a first stage, in the second stage will count the cells using the novel gradient circular hough transform (GCHT) method. the new GCHT method can segment touched cells and even overlapped cells.\\

Yan Kong et al \textsuperscript{\cite{Kong:20}} have developed a two-stage framework using parallel modified U-Nets together with seed guided water-mesh algorithm for automatic segmentation and yeast cells counting which is used to observe the living conditions and survival of yeast cells under experimental conditions. This method achieved a precision of over 96\% and a recall rate of 99.35\%.\\

Overton, Toyah and Tucker, Allan \textsuperscript{\cite{10.1007/978-3-030-44584-3_31}} have developed a method about segmentation and counting IDP (Internally Displaced people) and erythrocytes (red blood cells) using DO-U-Net (Dual Output U-Net) which outputs a segmentation mask and an edge mask then they substract them to get rid of the overlapping and the touching problem, the model trains on extremely small datasets (10 images) and gives a high segmentation accuracy, for the IDP they had 98.69\% for fixed resolution images and 94.66\% for scale-invariant satellite images.\\

\section{Conclusion}
\vspace{0.1in}
\hspace*{0.16in}
