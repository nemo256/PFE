\vspace*{0.2in}

\thispagestyle{empty}

\begin{center}
    {\color{Black} \rule{3in}{1.4mm} }\\
    \vspace{0.1in}
    \scshape{\fontsize{34}{46}{\bfseries{\color{Black}{Abstract}}}}
    \\
    \vspace{0.6in}
\end{center}
\cftaddtitleline{toc}{part}{\vspace{-0.12in}\color{Black}{Abstract}}{}
\hspace{\parindent}
\begin{changemargin}{0.9cm}{0.9cm}

Cell counting is a tedious task that would benefit greatly from automation. Accurate cell counting CBC (Complete Blood Count) provides essential quantitative information and plays a key role in biological research as well as industrial and biomedical applications. Unfortunately, the commonly used manual counting method is time consuming, poorly standardized and not reproducible. The task is made even more difficult by overlapping cells and poor imaging quality. In this paper we compare between two convolutional neural networks that will segment the cells as a first phase, then count them in the second phase using 3 diffrent algorithms (Watershed, Connected Component Labeling et Circle Hough Transform).

\end{changemargin}

\vspace{1in}

\begin{changemargin}{0.9cm}{0.9cm}
\hspace{-\parindent}
Keywords: CNN, cell segmentation, cell counting, convolutional neural networks
\end{changemargin}
