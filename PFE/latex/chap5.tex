\section{Introduction}
\vspace{0.2in}
\hspace*{0.16in}
In our case study, and from multiple articles, we can see that U-Net and Segnet models are dominating the field of cell segmentation.
In this chapter, we test out both U-Net and Segnet models, and analyse the results by comparing the two architectures.
We will also explore different machine learning algorithms for both preprocessing and postprocessing.

\section{Models}
\vspace{0.2in}
\hspace*{0.16in}
\lipsum[3-3]

\subsection{U-Net}
\subsubsection{Introduction}
\lipsum[2-2]

\subsection{Segnet}
\subsubsection{Introduction}
SegNet is a semantic segmentation model. This core trainable segmentation architecture consists of an encoder network, a corresponding decoder network followed by a pixel-wise classification layer. The architecture of the encoder network is topologically identical to the 13 convolutional layers in the VGG16 network. \textsuperscript{\cite{badrinarayanan2017segnet}}
