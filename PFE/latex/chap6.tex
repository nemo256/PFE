\section{Introduction}
\vspace{0.2in}
\hspace*{0.16in}
%TODO explain what we tested with steps 

\section{Used Tools}
\vspace{0.2in}
\hspace*{0.16in}


\subsection{TensorFlow}
In our work we used TensorFlow which is an open source library developed by Google Brain Team for Artificial Intelligence, it contains multiple pre-defined models and algorithms for Deep Learning and Machine Learning, tensorFlow can be used in multiple languages as python, C++, Java and JavaScript.\\
Tensorflow can be used to Create, Train, Deploy Models. And when we talk about complicated models we can use Keras.
Keras is a high level API for Neural Networks which helps with experiments on the models and extensibility.

\subsection{Colab}
Google Colab or Colaboratory is a free Jupyter notebook environment running on Google's cloud servers for machine learning training and research. This platform allows the user to leverage backend hardware such as GPUs and TPUs and train Machine Learning and Deep Learning models directly in the cloud. Without the need to install anything on our computer at the anything on our computer except a browser.
but it has some disadvantages where we have a limit on the GPU usage
and non presistant storage.

\subsection{PaperSpace}
Paperspace is a high-performance cloud computing and ML development platform for building, training and deploying machine learning models. It has a complete jupyter notebook environement which has a persistent storage and 6 hours limit for each execution.


\section{Experiments and Results}

\subsection{Red Blood Cells}
\subsubsection{DO-U-Net}
%TODO we will have a table which has accuracy and metrics for example 10 images and their real and predicted count with the 3 methods
%example of table columns
% Image Results(Metrics IOU ...) Real-Count CHT CCL Watershed

\subsection{DO-SegNet}
%TODO we will do the same thing as DO-UNET and at the end we draw a resume table wich will compare between both/ or we can do both in the same table 

\subsection{White Blood Cells}
%TODO Same
\subsubsection{DO-U-Net}
\subsubsection{DO-SegNet}


\subsection{Platlets}
%TODO Same
\subsubsection{DO-U-Net}
\subsubsection{DO-SegNet}

%TODO at the end we draw some tables that will contain the 3 

\subsection{}
\vspace{0.2in}
\hspace*{0.16in}
